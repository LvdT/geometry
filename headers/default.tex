\usepackage{polyglossia}
\setmainlanguage[variant=british]{english}
\usepackage[none]{hyphenat}

\usepackage{geometry}
\geometry{left=20mm,right=20mm,top=20mm,bottom=20mm}

\usepackage{amssymb,amsmath,amsthm}
\usepackage{booktabs,multirow}
\usepackage{graphicx}
\usepackage{enumerate,enumitem}

\usepackage{caption}
\usepackage{subcaption}
\usepackage[hidelinks,pdfusetitle]{hyperref}
\usepackage[figure]{hypcap}
\usepackage{float}

\usepackage{cool}
\Style{DSymb={\mathrm{d}},IntegrateDifferentialDSymb={\mathrm{d}}}

\usepackage[math-style=ISO,bold-style=ISO]{unicode-math}
\AtBeginDocument{%
  \let\phi\varphi
  \let\epsilon\varepsilon
}
\defaultfontfeatures{Ligatures=TeX,ExternalLocation=fonts/,Extension=.otf}
\setmathfont{XITS Math}
\setmathfont[range={\mathcal,\mathbfcal},StylisticSet=1]{XITSMath}
\setmainfont[
  UprightFont=*R,
  ItalicFont=*I,
  BoldFont=*B,
  BoldItalicFont=*BI
]{LibertinusSerif}
\setsansfont[
  UprightFont=*R,
  ItalicFont=*I,
  BoldFont=*B
]{LibertinusSans}
\setmonofont[
  UprightFont=*R
]{LibertinusMono}
