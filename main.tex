\documentclass[12pt]{scrartcl}
\title{Geometry}
\author{I. Saltini}
\date{}
\setlength{\parindent}{0pt}

\usepackage{polyglossia}
\setmainlanguage[variant=british]{english}
\usepackage[none]{hyphenat}

\usepackage{geometry}
\geometry{left=20mm,right=20mm,top=20mm,bottom=20mm}

\usepackage{amssymb,amsmath,amsthm}
\usepackage{booktabs,multirow}
\usepackage{graphicx}
\usepackage{enumerate,enumitem}

\usepackage{caption}
\usepackage{subcaption}
\usepackage[hidelinks,pdfusetitle]{hyperref}
\usepackage[figure]{hypcap}
\usepackage{float}

\usepackage{cool}
\Style{DSymb={\mathrm{d}},IntegrateDifferentialDSymb={\mathrm{d}}}

\usepackage[math-style=ISO,bold-style=ISO]{unicode-math}
\defaultfontfeatures{Ligatures=TeX,ExternalLocation=fonts/,Extension=.otf}
\setmathfont{XITS Math}
\setmathfont[range={\mathcal,\mathbfcal},StylisticSet=1]{XITSMath}
\setmainfont[
  UprightFont=*R,
  ItalicFont=*I,
  BoldFont=*B,
  BoldItalicFont=*BI
]{LibertinusSerif}
\setsansfont[
  UprightFont=*R,
  ItalicFont=*I,
  BoldFont=*B
]{LibertinusSans}
\setmonofont[
  UprightFont=*R
]{LibertinusMono}

\usepackage[
  list-units=single,
  range-units=single,
  multi-part-units=single,
  exponent-product=\cdot,
  per-mode=fraction,
  retain-explicit-plus=true
]{siunitx}
\DeclareSIUnit{\atp}{at.\percent}
\DeclareSIUnit{\magn}{\ensuremath{\times}}
\DeclareSIUnit{\rpm}{rpm}
\DeclareSIUnit{\nit}{nt}
\DeclareSIUnit{\talbot}{Tb}

%vectors
\NewDocumentCommand\vv{m}{\symbf{#1}}
\NewDocumentCommand\vs{m}{\symbf{u}_{#1}}
\NewDocumentCommand\norm{m}{\|#1\|}

%matrices
\NewDocumentCommand\idn{}{\mathbb{1}}
\NewDocumentCommand\inv{m}{{#1}^{-1}}
\NewDocumentCommand\tr{m}{{#1}^t}
\NewDocumentCommand\dt{m}{\left|#1\right|}

%bracketing
\NewDocumentCommand\rb{m}{\left(#1\right)}

%set theory
\NewDocumentCommand\pw{m}{\mathcal{P}\rb{#1}}

% \usepackage{csquotes}
\usepackage[sorting=none,isbn=true,url=false,doi=false,backend=biber]{biblatex}

% \usepackage[usenames,dvipsnames]{xcolor}

\usepackage{tikz}
\usetikzlibrary{
  positioning,
  shapes,
  shadows,
  arrows,
  fit,
  decorations,
  patterns,
  mindmap
}

\usepackage{readarray}
\usepackage{ifthen}
\usepackage{pgfplots}
\usepackage{chemfig}

% \usepackage[most]{tcolorbox}
\newtcolorbox{examplebox}[1][0]{
  colback=MyPaleBlue,
  colframe=MyDarkBlue,
  colbacktitle=MyLiteBlue,enhanced,
  fonttitle=\bfseries,
  attach boxed title to top center={yshift=-2mm},
  title=#1
}


\begin{document}
\maketitle

\section{Topological space}

A topological space is defined as the pair \(\rb{T,\nu}\) where \(T\) is a set and \(\nu : T \to \pw{T}\) is a function, called \textbf{neighbourhood topology}, which has to satisfy the following axioms.
The elements of \(T\) are called \textbf{points} and, given a point \(x\) the elements of \(\nu(x)\) are called \textbf{neighbourhoods} of \(x\).

\begin{enumerate}[label=\textbf{TS.\arabic*}]
  \item A neighbourhood of a point must contain the point itself.
  \item All supersets of a neighbourhood of a point are neighbourhoods of that point.
  \item The intersection of two neighbourhoods of the same point is a neighbourhood of that point.
  \item Any neighbourhood \(n\) of a point must contain another neighbourhood \(m\) of the same point, such that \(n\) is a neighbourhood of all points in \(m\).
\end{enumerate}

We can restate the axioms in formal terms.
For brevity it is implied that \(x, y \in T\) and \(n, m \in \pw{T}\).

\begin{enumerate}[label=\textbf{TS.\arabic*}]
  \item \(\forall x \forall n : n \in \nu(x) \to x \in n\)
  \item \(\forall x \forall n \forall m : \rb{n \in \nu(x) \land n \subseteq m} \to m \in \nu(x)\)
  \item \(\forall x \forall n \forall m : \rb{n \in \nu(x) \land m \in \nu(x)} \to n \cap m \in \nu(x)\)
  \item \(\forall x \forall n \exists m \forall y : \rb{n \in \nu(x) \land m \subseteq n \land m \in \nu(x) \land y \in m} \to n \in \nu(y)\)
\end{enumerate}

If \(\rb{T,\nu}\) is a topological space, we say that \(u \in \pw{T}\) is \textbf{open} if it is a neighbourhood of all the points it contains.
Restating this in formal terms, \(u\) is open if:
%
\[\forall x : x \in u \to u \in \nu(x)\]
%
We say that \(u \in \pw{\mathfrak{T}}\) is \textbf{closed} if its complement \(\cmpl{u}\) is open.
It is important to remark that the two definitions are not mutually exclusive, a set could be both open and closed, in which case it’s usually called a clopen set.

\end{document}
