\documentclass[12pt]{scrartcl}
\title{Geometry}
\author{I. Saltini}
\date{}
\setlength{\parindent}{0pt}

\usepackage{polyglossia}
\setmainlanguage[variant=british]{english}
\usepackage[none]{hyphenat}

\usepackage{geometry}
\geometry{left=20mm,right=20mm,top=20mm,bottom=20mm}

\usepackage{amssymb,amsmath,amsthm}
\usepackage{booktabs,multirow}
\usepackage{graphicx}
\usepackage{enumerate,enumitem}

\usepackage{caption}
\usepackage{subcaption}
\usepackage[hidelinks,pdfusetitle]{hyperref}
\usepackage[figure]{hypcap}
\usepackage{float}

\usepackage{cool}
\Style{DSymb={\mathrm{d}},IntegrateDifferentialDSymb={\mathrm{d}}}

\usepackage[math-style=ISO,bold-style=ISO]{unicode-math}
\defaultfontfeatures{Ligatures=TeX,ExternalLocation=fonts/,Extension=.otf}
\setmathfont{XITS Math}
\setmathfont[range={\mathcal,\mathbfcal},StylisticSet=1]{XITSMath}
\setmainfont[
  UprightFont=*R,
  ItalicFont=*I,
  BoldFont=*B,
  BoldItalicFont=*BI
]{LibertinusSerif}
\setsansfont[
  UprightFont=*R,
  ItalicFont=*I,
  BoldFont=*B
]{LibertinusSans}
\setmonofont[
  UprightFont=*R
]{LibertinusMono}

%\usepackage[
  list-units=single,
  range-units=single,
  multi-part-units=single,
  exponent-product=\cdot,
  per-mode=fraction,
  retain-explicit-plus=true
]{siunitx}
\DeclareSIUnit{\atp}{at.\percent}
\DeclareSIUnit{\magn}{\ensuremath{\times}}
\DeclareSIUnit{\rpm}{rpm}
\DeclareSIUnit{\nit}{nt}
\DeclareSIUnit{\talbot}{Tb}

%vectors
\NewDocumentCommand\vv{m}{\symbf{#1}}
\NewDocumentCommand\vs{m}{\symbf{u}_{#1}}
\NewDocumentCommand\norm{m}{\|#1\|}

%matrices
\NewDocumentCommand\idn{}{\mathbb{1}}
\NewDocumentCommand\inv{m}{{#1}^{-1}}
\NewDocumentCommand\tr{m}{{#1}^t}
\NewDocumentCommand\dt{m}{\left|#1\right|}

%bracketing
\NewDocumentCommand\rb{m}{\left(#1\right)}

%set theory
\NewDocumentCommand\pw{m}{\mathcal{P}\rb{#1}}

\newtheorem{proposition}{Proposition}
\newtheorem{corollary}{Corollay}[proposition]
\renewenvironment{proof}{{\emph{Proof.}}}{}
\theoremstyle{definition}
\newtheorem*{notion}{Notion}
\newtheorem*{definition}{Definition}
\newtheorem{example}{Example}
\newtheorem{observation}{Observation}
\newtheorem{axiom}{Axiom}

% \usepackage{csquotes}
\usepackage[sorting=none,isbn=true,url=false,doi=false,backend=biber]{biblatex}

% \usepackage[usenames,dvipsnames]{xcolor}

\usepackage{tikz}
\usetikzlibrary{
  positioning,
  shapes,
  shadows,
  arrows,
  fit,
  decorations,
  patterns,
  mindmap
}

\usepackage{readarray}
\usepackage{ifthen}
\usepackage{pgfplots}
\usepackage{chemfig}

% \usepackage[most]{tcolorbox}
\newtcolorbox{examplebox}[1][0]{
  colback=MyPaleBlue,
  colframe=MyDarkBlue,
  colbacktitle=MyLiteBlue,enhanced,
  fonttitle=\bfseries,
  attach boxed title to top center={yshift=-2mm},
  title=#1
}


\begin{document}
\maketitle

\section{Topological space}
\begin{definition}
 A \textbf{neighbourhood topology} on a set \(\mathfrak{T}\) is a function \(\nu : \mathfrak{T} \to \pw{\mathfrak{T}}\) mapping \textbf{points} (elements of \(\mathfrak{T}\), from here on denoted by \(x\), \(y\) or \(z\)) to \textbf{neighbourhoods} (subsets of \(\mathfrak{T}\), from here on denoted by \(n\) or \(m\)). It must satisfy the following axioms.
 \begin{enumerate}[label=\roman*)]
   \item A neighbourhood of a point must contain the point itself.
   \[\forall x \forall n : n \in \nu(x) \to x \in n\]
   \item All supersets of a neighbourhood of a point are neighbourhoods of that point.
   \[\forall x \forall n \forall m : \rb{n \in \nu(x) \land n \subseteq m} \to m \in \nu(x)\]
   \item The intersection of two neighbourhoods of the same point is a neighbourhood of that point.
   \[\forall x \forall n \forall m : \rb{n \in \nu(x) \land m \in \nu(x)} \to n \cap m \in \nu(x)\]
   \item Any neighbourhood \(n\) of a point must contain another neighbourhood \(m\) of the same point, such that \(n\) is a neighbourhood of all points in \(m\).
   \[\forall x \forall n \exists m \forall y : \rb{n \in \nu(x) \land m \subseteq n \land m \in \nu(x) \land y \in m} \to n \in \nu(y)\]
 \end{enumerate}
\end{definition}
%
\begin{definition}
  A \textbf{topological space} \(\rb{\mathfrak{T},\nu}\) is a set \(\mathfrak{T}\) equipped with a neighbourhood topology \(\nu\).
\end{definition}
%
\begin{definition}
  Two points \(x, y \in \mathfrak{T}\) are said to be \textbf{topologically indistinguishable} if they have exactly the same neighbourhoods.
  \[x \equiv y \liff \forall n : n \in \nu(x) \liff n \in \nu(y)\]
\end{definition}
%
\begin{definition}
  Two points \(x, y \in \mathfrak{T}\) are said to be \textbf{topologically distinguishable} if they are not topologically indistinguishable, i.e. if there is at least one neighbourhood of one of the points that isn’t a neighbourhood of the other.
  \[x \nequiv y \liff \exists n : n \in \nu(x) \lxor n \in \nu(y)\]
\end{definition}
%
\begin{proposition}
  Topological indistinguishability is an equivalence relation, i.e. \((\mathfrak{T},\equiv)\) is a setoid.
\end{proposition}
\begin{proof}
  Trivial, the necessary properties are inherited from the biconditional.
\end{proof}
%
\begin{corollary}
  Topological distinguishability is an apartness relation, i.e. \((\mathfrak{T},\nequiv)\) is a constructive setoid.
\end{corollary}
\begin{proof}
  The dual of an equivalence relation is an apartness relation.
\end{proof}

\begin{definition}
  A point \(x\) is a \textbf{limit point} for \(u\) if each of its neighbourhoods also contains at least one point of \(u\) distinguishable from \(x\) itself.
  \[\mathrm{Lim}(x,u) \liff \forall n \exists y : n \in \nu(x) \lif \rb{y \in n \land y \in u \land x \nequiv y}\]
\end{definition}

\begin{definition}
  A point \(x\) is an \textbf{isolated point} in \(u\) if it has at least one neighbourhood that contains no other points in \(u\) except for those indistinguishable from \(x\) itself.
  \[\mathrm{Iso}(x,u) \liff \exists n \forall y : n \in \nu(x) \land \rb{y \in n \lif \rb{y \notin u \lxor x \equiv y}}\]
\end{definition}

\begin{proposition}
  All points in a set are either limit points or isolated points.
  \[x \in u \lif \mathrm{Lim}(x,u) \lxor \mathrm{Iso}(x,u)\]
\end{proposition}

\begin{definition}
  A point \(x\) is an \textbf{interior point} for \(u\) if it has at least one neighbourhood that is entirely contained in \(u\). The point must therefore belong to the set.
  \[\mathrm{Int}(x,u) \liff \exists n \forall y : \rb{n \in \nu(x) \land y \in n} \lif y \in u\]
\end{definition}

\begin{definition}
  A point \(x\) is a \textbf{boundary point} for \(u\) if each of its neighbourhoods contains at least one point belonging to \(u\) and at least one point not belonging to \(u\).
  \[\mathrm{Bnd}(x,u) \liff \forall n \exists y \exists z : n \in \nu(x) \land y \in n \land y \in u \land z \in n \land z \notin u\]
\end{definition}

\begin{definition}
  A point \(x\) is an \textbf{exterior point} for \(u\) if it has at least one neighbourhood that is entirely contained in \(u\). The point must therefore not belong to the set.
  \[\mathrm{Ext}(x,u) \liff \exists n \forall y : \rb{n \in \nu(x) \land y \in n} \lif y \notin u\]
\end{definition}

\begin{definition}
  \(u\) is \textbf{open} if it is a neighbourhood of all the points it contains.
  \[\mathrm{Open}(u) \liff \forall x : x \in u \to u \in \nu(x)\]
\end{definition}
%
\begin{definition}
  \(u\) is \textbf{closed} its complement \(\mathfrak{T} \setminus u\) is open.
\end{definition}
%
\begin{observation}
  Despite what the choice of terminology might suggest, the definitions of open and closed are not mutually exclusive. A set that is both open and closed is said to be \textbf{clopen}.
\end{observation}

\section{Curvature Tensor}

\begin{definition}
  The \textbf{Riemann curvature tensor} \({R^\rho}_{\sigma\mu\nu}\) and the \textbf{Cartan torsion tensor} \({T_{\mu\nu}^\lambda}\) are the tensors that satisfy the following equation for all vector fields \(\phi\).
  \[\comm{\nabla_\mu}{\nabla_\nu} \phi^\rho = {R^\rho}_{\sigma\mu\nu} \phi^\sigma + {T_{\mu\nu}}^\lambda \nabla_\lambda \phi^\rho\]
\end{definition}

\begin{proposition}
  In a holonomic basis the Riemann curvature tensor and Cartan torsion tensors have the following expressions in terms of the Christoffel symbols.
  \[{R^\rho}_{\sigma\mu\nu} =
  \partial_\mu \Chr{\rho}{\nu\sigma} -
  \partial_\nu \Chr{\rho}{\mu\sigma} +
  \Chr{\rho}{\mu\lambda} \Chr{\lambda}{\nu\sigma} -
  \Chr{\rho}{\nu\lambda} \Chr{\lambda}{\mu\sigma}\]
  \[{T_{\mu\nu}}^\lambda = \Chr{\lambda}{\mu\nu} - \Chr{\lambda}{\nu\mu}\]
\end{proposition}
\begin{proof} We begin by computing the action of one of the firt term of the commutator onto a component of \(\phi\), recalling that \(\nabla_\nu \phi^\rho\) is a type \((1,1)\) tensor.
  \begin{align*}
    \nabla_\mu \nabla_\nu \phi^\rho &= \partial_\mu \nabla_\nu \phi^\rho + \Chr{\rho}{\mu\lambda} \nabla_\nu \phi^\lambda - \Chr{\lambda}{\nu\mu} \nabla_\lambda \phi^\rho \\
    &=\partial_\mu \rb{\partial_\nu \phi^\rho + \Chr{\rho}{\nu\sigma} \phi^\sigma} + \Chr{\rho}{\mu\lambda} \rb{\partial_\nu \phi^\lambda + \Chr{\lambda}{\nu\sigma} \phi^\sigma} - \Chr{\lambda}{\nu\mu} \rb{\partial_\lambda \phi^\rho + \Chr{\rho}{\lambda\sigma} \phi^\sigma} \\
    &=\partial_\mu \partial_\nu \phi^\rho + \rb{\partial_\mu \Chr{\rho}{\nu\sigma}} \phi^\sigma + \Chr{\rho}{\nu\sigma} \partial_\mu \phi^\sigma + \Chr{\rho}{\mu\lambda} \partial_\nu \phi^\lambda + \Chr{\rho}{\mu\lambda} \Chr{\lambda}{\nu\sigma} \phi^\sigma - \Chr{\lambda}{\nu\mu} \partial_\lambda \phi^\rho - \Chr{\lambda}{\nu\mu} \Chr{\rho}{\lambda\sigma} \phi^\sigma
  \end{align*}
  In the fourth term, we replace the dummy index \(\lambda\) with \(\sigma\) in order to factor the expression.
  \begin{align*}
    \nabla_\mu \nabla_\nu \phi^\rho &= \partial_\mu \partial_\nu \phi^\rho + \rb{\partial_\mu \Chr{\rho}{\nu\sigma}} \phi^\sigma + \Chr{\rho}{\nu\sigma} \partial_\mu \phi^\sigma + \Chr{\rho}{\mu\sigma} \partial_\nu \phi^\sigma + \Chr{\rho}{\mu\lambda} \Chr{\lambda}{\nu\sigma} \phi^\sigma - \Chr{\lambda}{\nu\mu} \partial_\lambda \phi^\rho - \Chr{\lambda}{\nu\mu} \Chr{\rho}{\lambda\sigma} \phi^\sigma \\
    &= \rb{\partial_\mu \partial_\nu - \Chr{\lambda}{\nu\mu} \partial_\lambda} \phi^\rho + \rb{\partial_\mu \Chr{\rho}{\nu\sigma} + \Chr{\rho}{\nu\sigma} \partial_\mu + \Chr{\rho}{\mu\sigma} \partial_\nu + \Chr{\rho}{\mu\lambda} \Chr{\lambda}{\nu\sigma} - \Chr{\lambda}{\nu\mu} \Chr{\rho}{\lambda\sigma}} \phi^\sigma
  \end{align*}
  We obtain the second term of the commutator by simply switching \(\nu\) with \(\mu\).
  \begin{equation*}
    \nabla_\nu \nabla_\mu \phi^\rho = \rb{\partial_\nu \partial_\mu - \Chr{\lambda}{\mu\nu} \partial_\lambda} \phi^\rho + \rb{\partial_\nu \Chr{\rho}{\mu\sigma} + \Chr{\rho}{\mu\sigma} \partial_\nu + \Chr{\rho}{\nu\sigma} \partial_\mu + \Chr{\rho}{\nu\lambda} \Chr{\lambda}{\mu\sigma} - \Chr{\lambda}{\mu\nu} \Chr{\rho}{\lambda\sigma}} \phi^\sigma
  \end{equation*}
  We now take the difference of the two terms, remembering that \(\comm{\partial_\mu}{\partial_\nu} = 0\).
  \begin{align*}
    \rb{\nabla_\mu \nabla_\nu - \nabla_\nu \nabla_\mu} \phi^\rho &= \rb{\Chr{\lambda}{\mu\nu} - \Chr{\lambda}{\nu\mu}} \partial_\lambda \phi^\rho + \rb{\partial_\mu \Chr{\rho}{\nu\sigma} - \partial_\nu \Chr{\rho}{\mu\sigma} + \Chr{\rho}{\mu\lambda} \Chr{\lambda}{\nu\sigma} - \Chr{\rho}{\nu\lambda} \Chr{\lambda}{\mu\sigma}} \phi^\sigma + \rb{\Chr{\lambda}{\mu\nu} - \Chr{\lambda}{\nu\mu}}\Chr{\rho}{\lambda\sigma} \phi^\sigma \\
    &= \rb{\partial_\mu \Chr{\rho}{\nu\sigma} - \partial_\nu \Chr{\rho}{\mu\sigma} + \Chr{\rho}{\mu\lambda} \Chr{\lambda}{\nu\sigma} - \Chr{\rho}{\nu\lambda} \Chr{\lambda}{\mu\sigma}} \phi^\sigma + \rb{\Chr{\lambda}{\mu\nu} - \Chr{\lambda}{\nu\mu}} \nabla_\lambda \phi^\rho
  \end{align*}
  Where we recognise the expressions for \({R^\rho}_{\sigma\mu\nu}\) and \({T_{\mu\nu}}^\lambda\).
\end{proof}

\end{document}
